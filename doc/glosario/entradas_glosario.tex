\newglossaryentry{git}
{
name={git},
description={Software de control de versiones.}
}
\newglossaryentry{DevOps}
{
name={DevOps},
description={Acrónimo de desarrollo y operaciones, se basa en la integración entre el desarrollo y la administración de sistemas.}
}
\newglossaryentry{microciclo}
{
name={microciclo},
description={Cada uno de los ciclos que forman un ciclo o periodo de entrenamiento.}
}
\newglossaryentry{backend}
{
name={backend},
description={Parte del software que interactúa y procesa las peticiones del Frontend.}
}
\newglossaryentry{frontend}
{
name={frontend},
description={Parte del software con la que interactúa el usuario.}
}

\newglossaryentry{API}
{
name={API},
description={Abreviación de interfaz de programación de aplicaciones o Application Programming Interface. }
}

\newglossaryentry{Integracion continua}
{
name={Integración continua},
description={Práctica de añadir los cambios de forma continua, con el fin de encontrar errores lo antes posible.}
}

\newglossaryentry{Despliegue continuo}
{
name={Despliegue continuo},
description={Práctica de desplegar los cambios de forma continua.}
}

\newglossaryentry{Test de cobertura}
{
name={Test de cobertura},
description={Test que proporciona una medida indicando la cantidad de código que ha sido probada.}
}

\newglossaryentry{Test unitarios}
{
name={Test unitarios},
description={Test que prueba una funcionalidad.}
}

\newglossaryentry{test A/B}
{
name={test A/B},
description={Test que consiste en mostrar dos aplicaciones con diferencias con el fin de ver cual agrada más a los usuarios.}
}

\newglossaryentry{provisionamiento}
{
name={provisionamiento},
description={Proceso de configurar una maquina con las dependencias necesarias.}
}

\newglossaryentry{NPM}
{
name={NPM},
description={Es el gestor de paquetes por defecto de Node.js}
}

\newglossaryentry{material design}
{
name={material design},
description={Estilo de diseño desarrollado por Google.}
}

\newglossaryentry{push}
{
name={push},
description={Acción se subir cambios a un repositorio de Git.}
}

\newglossaryentry{No-SQL}
{
name={No-SQL},
description={Tipo de base de datos que destaca por no hacer uso de SQL}
}

\newglossaryentry{middleware}
{
name={middleware},
description={Pieza de software que actúa como intermediaria en una comunicación.}
}

\newglossaryentry{endpoint}
{
name={endpoint},
description={Ruta a través de la cual podemos comunicarnos con un servidor.}
}

\newglossaryentry{pull request}
{
name={pull request},
description={Petición de fusionar los cambios de una rama o un fork con otra rama o repositorio.}
}

\newglossaryentry{merge}
{
name={merge},
description={Acción de mezclar código.}
}

\newglossaryentry{SPA}
{
name={SPA},
description={Single Page Application, pagina web que da la sensación de ser solo una.}
}

\newglossaryentry{RESTful}
{
name={RESTful},
description={Set de principios para establecer la arquitectura de un servicio web.}
}

\newglossaryentry{logging}
{
name={logging},
description={definición}
}

\newglossaryentry{BSON}
{
name={BSON},
description={definición}
}

\newglossaryentry{JSON}
{
name={JSON},
description={definición}
}

\newglossaryentry{JSX}
{
name={JSX},
description={definición}
}

\newglossaryentry{ODM}
{
name={ODM},
description={definición}
}

\newglossaryentry{driver}
{
name={driver},
description={definición}
}

\newglossaryentry{pseudojoins}
{
name={pseudojoins},
description={definición}
}

\newglossaryentry{orquestacion}
{
name={orquestación},
description={definición}
}

\newglossaryentry{bundle}
{
name={bundle},
description={definición}
}

\newglossaryentry{proxy inverso}
{
name={proxy inverso},
description={definición}
}

\newglossaryentry{hooks}
{
name={hooks},
description={definición}
}

\newglossaryentry{linter}
{
name={linter},
description={definición}
}

\newglossaryentry{PaaS}
{
name={PaaS},
description={definición}
}

\newglossaryentry{build}
{
name={build},
description={definición}
}

\newglossaryentry{Full Stack}
{
name={Full Stack},
description={definición}
}
