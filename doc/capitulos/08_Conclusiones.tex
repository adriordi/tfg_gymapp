\chapter{Conclusiones}

Como reflexión sobre la ejecución del proyecto, es importante remarcar tal como se puede ver en la bibliografía la importancia del uso de la documentación oficial de cada paquete en lugar de tutoriales no oficiales. Además, en caso de usar un tutorial, hay que remarcar la importancia de que haya sido desarrollado por alguna persona o empresa con buena reputación en el tema, para así garantizar que se siguen unos patrones de calidad.


Otro punto importante en la ejecución del proyecto, es la temporización que se ha ajustado al diagrama de Gantt mostrado en el Capítulo 3. No obstante, la última fase ha sufrido un pequeño retraso debido a la carga laboral a la que me he visto sometido, pero al haber realizado la planificación con un tiempo realista, se ha podido subsanar y el desarrollo se ha completado con éxito, como se muestra en el Capítulo 7.

Como se comenta en párrafos posteriores, existen formas de seguir extendiendo la aplicación, y la oportunidad de explotar los datos para la creación de nuevas funcionalidades y características más innovadoras.

Tal como se lee a continuación y como se muestra a lo largo del documento, ha sido necesaria la aplicación de conocimientos en diversos ámbitos: las bases de datos, las redes de comunicación, la infraestructura, la programación orientada a objetos, programación asíncrona, etc.

Para la aplicación de los conocimientos anteriormente mencionados ha sido necesaria la selección de ciertas herramientas, como se muestra en el Capítulo 6 (en el que se profundiza en la implementación) la selección ha sido especialmente compleja debido a la cantidad de soluciones que se ofrecen para la programación web, tanto en el Backend como en el Frontend.

También, un punto destacable como se ha mencionado a lo largo del documento y especialmente en la sección de implementación, es que se ha usado la integración continua, el desarrollo basado en pruebas y herramientas como Greenkeeper y SonarQube para mantener el código actualizado y libre de malas prácticas. Debido a esto, ha sido necesaria una revisión constante del proyecto a lo largo del tiempo, de forma que se han corregido errores, vulnerabilidades e introducido mejoras.

Como sabemos, Internet pone a disposición de todo el mundo la aplicación hay que valorar algunos aspectos éticos y sociales, como puede ser el hecho de que los datos almacenados son completamente anónimos para garantizar la privacidad de los usuarios y además se advierte que una herramienta de apoyo, y que una serie de datos y gráficas no sustituye a un profesional. Desde el punto de vista social, se ha tratado de hacer una interfaz minimalista y sencilla para facilitar su uso, con botones grandes, iconos y colores que puedan ser fácilmente identificable por todas las personas, y en caso de no serlo se les proporciona la correspondiente leyenda.


Para finalizar y desde un punto de vista mas personal, tengo que destacar que tras la realización de este proyecto, he adquirido grandes conocimientos en programación asíncrona, debido al comportamiento de Node y de las peticiones HTTP. Siendo concreto, debido a al funcionamiento de JavaScript, he aprendido a trabajar con promesas y las correspondientes funciones para crearlas y manejarlas, como pueden ser then, Async/Await... Además de eso he conseguido una comprensión más profunda del funcionamiento HTTP, por el desarrollo de la API, de las bases de datos No-SQL y el funcionamiento del DOM y las ventajas que aporta React en su manipulación.

Cabe destacar que debido a la gran cantidad de herramientas utilizadas, la puesta en marcha del desarrollo ha sido un tanto lenta, que debido a la naturaleza continua del desarrollo, las fases análisis y planificación han sido algo complejas, pero esto se compensa debido a la ayuda que dichas herramientas proporcionan para crear un código de calidad, modular y reutilizable.

El futuro del proyecto tiene dos claros frentes, en primer lugar mejorar los test y en segundo lugar, hacer uso de los datos almacenados.
\begin{itemize}
  \item \textbf{Mejorar los tests: } El primer objetivo a cumplir es añadir tests para el frontend y mejorar los ya existentes. Además se buscará incrementar la cobertura lo máximo posible, pero sin desperdiciar mucho tiempo en ello, ya que lo realmente importante es tener unos test de calidad, se puede tener una cobertura del 100\% y no por ello obtener un mejor código, con lo cual tampoco deberemos invertir más recursos de los necesarios en mejorar esa cifra. Por ello, lo importante será una gran cobertura de funcionalidades, pero no necesariamente de líneas de código.
  \item \textbf{Explotación de datos: } El siguiente paso lógico en el desarrollo, es hacer un uso intensivo de los datos almacenados, por un lado mostrando una pestaña de análisis con funciones más avanzadas, y por otro lado, con ayuda de personas con conocimientos más avanzados en fisiología del ejercicio, tratar de proporcionar información útil a los usuarios, detectar posibles anomalías y mostrar las posibles causas de ellas así como algunos consejos para afrontarlas.
\end{itemize}

Como nota final, diría que en caso de tener que volver a hacer un proyecto con React, aunque es mucho más didáctico, personalizable y potente llevar a a cabo toda la configuración de forma manual, haría uso de create-react-app, una herramienta de Facebook que genera automáticamente un entorno para el desarrollo de aplicaciones de React con una configuración bastante buena y que queda completamente oculta, liberándonos de los problemas derivados de esta y de horas de trabajo de configuración de herramientas. Además, probablemente también tomaría la decisión de cambiar GitHub por GitLab, ya que el segundo incluye herramientas de CI/CD integradas entre otras funciones.

Como conclusión me gustaría remarcar que el proyecto queda completamente libre, y que espero sea de utilidad a personas que quieran iniciarse en el desarrollo \gls{Full Stack}, así como en el uso de herramientas para integración continua, despliegue continuo, verificación de calidad del código, infraestructura virtual.
