\chapter{Conclusiones}

Tras la realización de este proyecto, he adquirido grandes conocimientos en programación asíncrona, debido al comportamiento de Node y de las peticiones HTTP. Siendo concreto, debido a al funcionamiento de JavaScript, he aprendido a trabajar con promesas y las correspondientes funciones para crearlas y manejarlas, como pueden ser them, Async/Await... Además de eso he conseguido una comprensión más profunda del funcionamiento HTTP, por el desarrollo de la API, de las bases de datos No-SQL y el funcionamiento del DOM y las ventajas que aporta React en su manipulación.

Cabe destacar que debido a la gran cantidad de herramientas utilizadas, la puesta en marcha del desarrollo ha sido un tanto lenta, que debido a la naturaleza continua del desarrollo, las fases análisis y planificación han sido algo complejas, pero esto se compensa debido a la ayuda que dichas herramientas proporcionan para crear un código de calidad, modular y reutilizable.

El futuro del proyecto tiene dos claros frentes, en primer lugar mejorar los test y en segundo lugar, hacer uso de los datos almacenados.
\begin{itemize}
  \item \textbf{Mejorar los tests: } El primer objetivo a complir es añadir tests para el frontend y mejorar los ya existentes. Además se buscará incrementar la cobertura lo maximo posible, pero sin desperdiciar mucho tiempo en ello, ya que lo realmente importante es tener uns test de calidad, se puede tener una cobertura del 100\% y no por ello obtener un mejor código, con lo cual tampoco deberemos invertir más recursos de los necesarios en mejorar esa cifra. Por ello, lo importante será una gran cobertura de funcionalidades, pero no necesariamente de líneas de código.
  \item \textbf{Explotación de datos: } El siguiente paso lógico en el desarrollo, es hacer un uso intensivo de los datos almacenados, por un lado mostrando una pestaña de análisis con funciones más avanzadas, y por otro lado, con ayuda de personas con conocimientos más avanzados en fisiología del ejercicio, tratar de proporcionar información útil a los usuarios, detectar posibles anomalías y mostrar las posibles causas de ellas así como algunos consejos para afrontarlas.
\end{itemize}

Como conclusión me gustaría remarcar que el proyecto queda completamente libre, y que espero sea de utilidad a personas que quieran iniciarse en el desarrollo Full Stack, así como en el uso de herramientas para integración continua, despliegue continuo, verificación de calidad del código, infraestructura virtual...


