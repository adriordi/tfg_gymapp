\chapter{Introducción}
En la actualidad el levantamiento de pesas es un tipo de entrenamiento ampliamente realizado, debido a que se conoce que puede aportar grandes beneficios a la salud, aunque si es realizado sin precaución se pueden producir lesiones, ya sea por una progresión mal planificada, una mala técnica de levantamiento, una mala nutrición o un déficit de descanso. Una de las tareas más sencillas para evitar lesiones, es observar la evolución de la progresión, buscando estancamientos o progresos demasiado rápidos. Por otro lado una tarea a realizar algo más compleja es el seguimiento de la carga aplicada por sesión de entrenamiento y grupo muscular.

Durante años practicando deporte he observado que la mayoría de deportistas amateur, se centran en la nutrición y la técnica, pero no prestan suficiente atención a la progresión, limitándose a cargar el máximo posible, o a la generación de desequilibrios musculares. 

En mi experiencia, este seguimiento no se realiza debido al desconocimiento la importancia de esta tarea, al desconocimiento de las progresiones basadas en ciclos y \glspl{microciclo}, a la complejidad de saber los músculos implicados en cada ejercicio, o simplemente a la pereza de llevar a cabo dicho seguimiento.

Debido a esas observaciones me propuse realizar una aplicación Web que de forma sencilla permite monitorizar el progreso y el uso de los diferentes grupos musculares, con lo cual, proporcionar una serie de datos con los que ayudar a un usuario a obtener sus propias conclusiones.

Dicho esto, cabe destacar que la aplicación simplemente proporcionará datos objetivos sobre la progresión y los grupos musculares usados, y que de ningún modo proporciona ningún consejo, esto se debe a que la planificación de un entrenamiento conlleva una gran complejidad, ya que cada individuo tiene sus propias características y esta tarea debe ser realizada por una profesional dada su implicación en la salud.

Para la implementación de este proyecto se llevará a cabo una metodología \gls{DevOps}, que quedará detallada en los capítulos 5 (Diseño) en el cual se explicaran las decisiones tomadas y 6 (Implementación) en el cual se mostrará como se ha llevado a cabo el desarrollo.

Por otro lado en los próximos tres capítulos ( Especificación de requisitos, Planificación y Análisis ), se mostrará que requisitos debe cumplir la aplicación para que pueda llevar a cabo su cometido, como se planificó el desarrollo del software a lo largo del tiempo y un análisis detallado de todo lo anterior.

Finalmente existen dos capítulos adicionales, en primer lugar el 7, en el que detallaremos los tests realizados sobre el software, tanto sobre la calidad del código, como sobre el buen funcionamiento de la aplicaciónes, tras esto se incluye un último capítulo donde se abordarán las conclusiones y la evolución futura de la aplicación.
