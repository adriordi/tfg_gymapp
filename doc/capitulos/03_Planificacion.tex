\chapter{Planificación}

\section {Fases y entregas}
\subsection {Fases}

Este proyecto estara compuesto por seis fases, combrando un gran peso en tiempo tres de ellas, la Analisis y diseño, la Implementación ( aunque se parta de una base de codigo iniciada con anterioridad) y la de pruebas, todas las fases serán las siguientes:

\begin{itemize}
  \item Fase 1: Especificaciones del proyecto.
  \item Fase 2: Planificación.
  \item Fase 3: Analisis y diseño.
  \item Fase 4: Implementacion.
  \item Fase 5: Pruebas.
  \item Fase 6: Documentación.
\end{itemize}

\subsection {Lista de actividades}
A continuación se muestra las actividades a desarollar en cada fase, y una estimación del tiempo requerido para poder cumplimentarlas.

\begin{itemize}
  \item Fase 1: Especificaciones del proyecto.
    \begin{enumerate}
      \item Determinación objetivos.
      \item Determinación requisistos.
    \end{enumerate}
  \item Fase 2: Planificación.
    \begin{enumerate}
      \item Lista de actividades
      \item Entrevistas
      \item Presupuesto
      \item Temporización
    \end{enumerate}
  \item Fase 3: Analisis y diseño.
    \begin{enumerate}
      \item Analisis de requisitos
      \item Diagramas
      \item Metodologia de desarollo.
      \item Descripción estructural
    \end{enumerate}
  \item Fase 4: Implementacion.
    \begin{enumerate}
      \item TODO
    \end{enumerate}
  \item Fase 5: Pruebas.
    \begin{enumerate}
      \item TODO
    \end{enumerate}
  \item Fase 6: Documentación.
    \begin{enumerate}
      \item Documentación API
      \item Documentación FrontEnd
      \item Documentación Proyecto
    \end{enumerate}
\end{itemize}


\section {Entrevistas}
Para poder alacanzar una interfaz comoda y de calidad, se ha realizado una serie de preguntas un test A/B a varios usuarios, de forma que usando el feedback proporcionado se han tomado las decisiones para la forma en la que se introducen y se muestran los datos.

\section {Presupuestos}

Para este proyecto se ha utilizado software libre y versiones gratuitas de ciertas plataformas como Auth0, TravisCI, GitHub, Greenkeper, SonnarQube, MongoLab, PaperTrail y Heroku entre otras. Por ello el coste en licencias en infraestructura es nulo. Por otro lado se han realizado despliegues en Google Cloud, que aunque se ha realizado haciendo uso de creditos gratuitos, si que conllevaria un gasto si se optara por mantener la aplicación en dicha plataforma.




